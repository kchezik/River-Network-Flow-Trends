\documentclass[12pt]{article}
\usepackage{geometry} 
\geometry{letterpaper}
\usepackage{graphicx}
%\usepackage{pdfpages}
%\usepackage{lscape}
%\usepackage{pdflscape}
%\usepackage{setspace}
%\usepackage{booktabs}
\usepackage{amsmath}
\usepackage{upgreek}
 

\title{}
\author{}
\date{}


\begin{document}
%\begin{spacing}{1.1}
%\doublespacing
%\onehalfspacing  
\maketitle
%\tableofcontents
%\thispagestyle{empty}
We calculated the probability that we were observing attenuation as the percentage of simulated $\hat{\updelta}$ that were greater than the observed $\hat{\updelta}$ (a negative $\hat{\updelta}$ indicates decreasing variance with increasing area, Fig.~3). We calculated attenuation strength as the ratio between the standard deviation at the smallest and largest  watersheds as defined by the exponential variance function  (i.e.,\ $\sqrt{f(\mathrm{area}_1)} / \sqrt{f(\mathrm{area}_2)}$). To compare the degree of attenuation observed with the null-simulated attenuation, we compared this ratio with the same ratio calculated from each of the null simulations (i.e.,\ $\left( \sqrt{f(\mathrm{area}_1)} / \sqrt{f(\mathrm{area}_2)} \right) / \left( \sqrt{f(\mathrm{area}_{\mathrm{null} 1})} / \sqrt{f(\mathrm{area}_{\mathrm{null} 2})} \right)$). Finally, to estimate how the day-of-year to half-annual-flow was changing, we calculated the steepest point of each site's logistic curve and multiplied this slope value by 365 to revert from our 0--1 scale to our original annual range of 0--365. To get a decadal rate we multiplied these results by 3.8 (number of decades in 38 years). 
\textit{
Could keep this sentence in or just leave it out. This is just a shortcut to calculate the first derivative. Maybe we don't need to describe the shortcut: 
}
We estimated the steepest part of the logistic curve using the \textit{divide by four} ruleADD REF HERE (dividing the coefficient by four equals the first derivative of the logistic curve at its steepest point).



%\end{spacing}
\end{document}

